\section{Generalità sulla redazione di progetti di sistemazione idraulico-forestale}
In questa sezione si andrà ad esporre in maniera generale un esempio di relazione di studio ambientale, da allegare al progetto di sistemazione idraulico-forestale.\\
Generalmente, l'indice del documento è formato da:
\begin{enumerate}
    \item premessa: vengono descritti gli obiettivi del progetto e la verifica alla valutazione di impatto ambientale;
    \item inquadramento territoriale: descrizione generale ed informativa del territorio circostante all'opera;
    \item quadro programmatico: quadro d'insieme delle normative ambientali e programmatiche presenti nel territorio di studio. I riferimenti normativi principalmente richiamati sono: Normativa Rete Natura 2000, Piano di Tutela delle Acque (PTA), Piano di Gestione del Rischio Alluvioni (PGRA), Pianificazione Territoriale (piano regionale, provinciale, o comunale), vincoli paesaggistici e ambientali (es. D.Lgs. 152/2006), vincolo idrogeologico (Legge Serpieri);
    \item quadro progettuale: esposizione delle analisi progettuali alternative, dello stato di fatto, del progetto strutturale dell'opera, dei volumi/portate in transito e dei lavori programmatici di manutenzione;
    \item quadro ambientale: vengono riportate le caratteristiche atmosferiche, ambientali, geologiche, di qualità dell'acqua, di flora/fauna e paesaggistiche;
    \item valutazione degli impatti e mitigazioni: vengono ipotizzati gli impatti negativi, generati dalla presenza dell'opera, nei confronti del suolo, dell'atmosfera, della flora, della fauna e del paesaggio. Per ogni impatto negativo vengono proposte delle azioni di mitigazione degli effetti;
    \item compensazioni: si espongono le opere o le attività di compensazione a favore della comunità, delle istituzioni o del territorio;
    \item considerazioni conclusive: in questa parte di relazione avviene una sintesi di tutto il documento di analisi ambientale da allegare alla costruzione dell'opera;
    \item bibliografia: sono riportati tutti i riferimenti normativi, tecnici o accademici con cui si è riusciti a redigere il documento.
\end{enumerate}
