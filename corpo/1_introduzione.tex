\section{Introduzione}
In questa relazione si andrà ad esporre i procedimenti necessari per l'analisi e la progettazione di un intervento di sistemazione e consolidamento idraulico, in un bacino idrografico della Provincia di Trento (Valle dei Mocheni).\\
I primi capitoli andranno ad analizzare ed esporre le principali caratteristiche del bacino idrografico dello studio.\\
Successivamente, verrà svolta la regolarizzazione dei dati pluviometrici secondo Gumbel, per ricavare la Linea Segnalatrice di Probabilità Pluviometrica (LSPP), in modo sia grafico che analitico.\\ 
Avendo ottenuto tali dati di input nel sistema, la relazione continuerà con l'analisi della trasformazione in deflusso delle piogge e del loro passaggio alla sezione di chiusura. Essendoci molti metodi in letteratura, verranno riportati i più utilizzati per ogni categoria.\\
Dopo aver stimato la componente idraulica agente nell'area di studio, verranno esposti i calcoli per ottenere la pendenza di correzione del tratto di studio e per il volume di materiale solido in movimento.\\
Infine, verrà svolto il dimensionamento (idraulico e statico) dell'opera di consolidamento tipo da costruire in alveo e della relativa controbriglia da ereggere a valle della serie.
