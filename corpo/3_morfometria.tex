\section{Morfometria del bacino}
In questa sezione si andrà ad analizzare i principali parametri quantitativi che è possibile applicare per valutare le caratteristiche morfometriche del bacino.
\subsection{Rapporto di biforcazione e prima legge di Horton}
Secondo la prima legge di Horton, il rapporto tra la numerosità dell'ordine precedente e la numerosità dell'ordine considerato (detto ``rapporto di biforcazione parziale")
\begin{equation}
    R_u = \frac{N_{u-1}}{N_u}
\end{equation}
è statisticamente costante, e regolata dalla funzione:
\begin{equation}
    N_u= \bar{R}_b ^{(k-u)}
\end{equation}
Dove: 
\begin{itemize}
    \item $R_b$ è la media tra i rapporti di biforcazione parziali;
    \item $k$ è l'ordine del bacino, nel caso di quello preso in considerazione il valore è 3;
    \item $u$ è l'ordine del tratto di reticolo considerato.
\end{itemize}
La prima legge di Horton inoltre, evidenzia come all'aumentare dell'ordine dei tratti, la lunghezza dei segmenti e le aree dei sottobacini aumentino, mentre cala la loro numerosità.\\
\paragraph{Reticolo idrografico, secondo la CTP} Nel caso del reticolo idrografico della CTP preso da noi in esame, i parametri sono: 

\begin{table}[H] \centering
    \caption{\textcolor{red}{Caratteristiche di biforcazione del reticolo idrografico ed applicazione della prima legge di Horton, secondo i dati ricavati dalla CTP.}}
    \label{tab:rapp_biforcazione_1_horton}
    \begin{tabular}{ cccc }
\toprule
    Ordine u & Segmenti Nu & Rapp. di biforcazione Rb & Nu (prima legge di Horton) \\
\midrule    
    1        & 5           &   /                     & 5.1                        \\
    2        & 2           & 2.5                      & 2.3                        \\
    3        & 1           & 2.0                      & 1.0                       \\
\bottomrule    
\end{tabular}
\end{table}
Il rapporto di biforcazione medio $\bar{R}_b$, secondo il ragionamento svolto sulla CTP è pari a \num{2.3}.\\
Interpolando i valori degli ordini dei segmenti, con la loro numerosità e con il parametro ricavato dalla prima legge di Horton, si ottiene un grafico caratteristico:
\begin{figure}[H]\centering
    \includegraphics[scale=.75]{immagini/legge_horton.png}
    \caption{Relazione tra l'ordine del tratto, la sua numerosità e la funzione di Horton.}
    \label{legge_horton}
\end{figure}

\paragraph{Reticolo idrografico, secondo il DTM} Nel caso dello studio del reticolo idrografico mediante l'utilizzo di Qgis e del relativo file DTM, i parametri estratti sono i seguenti:
\begin{table}[H] \centering
    \caption{\textcolor{red}{Caratteristiche di biforcazione del reticolo idrografico ed applicazione della prima legge di Horton, secondo i dati ricavati dal DTM.}}
    \label{tab:rapp_biforcazione_2_horton}
    \begin{tabular}{ cccc }
\toprule
    Ordine u & Segmenti Nu & Rapp. di biforcazione Rb & Nu (prima legge di Horton) \\
\midrule    
    1        & 12          &   /        & 12.3                        \\
    2        & 3           & 4.0       & 3.5                        \\
    3        & 1           & 3.0      & 1.0                       \\
\bottomrule    
\end{tabular}
\end{table}

Il rapporto di biforcazione medio $\bar{R}_b$, secondo il ragionamento svolto sul DTM è pari a \num{3.5}.\\
Interpolando i valori degli ordini dei segmenti, con la loro numerosità e con il parametro ricavato dalla prima legge di Horton, si ottiene un grafico caratteristico:
\begin{figure}[H]\centering
    \includegraphics[scale=.75]{immagini/legge_horton_qgis.png}
    \caption{Relazione tra l'ordine del tratto, la sua numerosità e la funzione di Horton.}
    \label{legge_horton_qgis}
\end{figure}

\vspace{1cm}
In entrambi i casi, la prima legge di Horton viene soddisfatta, poichè la funzione del grafico interpola correttamente i dati reali del reticolo idrografico.\\
Lo scostamento dei dati nei due casi è dovuto alla differente estrazione del reticolo idrografico nei due metodi.

\subsection{Coefficiente di forma di Gravelius}
Il coefficiente di forma di Gravelius è un parametro che indica quanto la forma di un bacino sia compatta.\\
Questo indicatore confronta le misure di area e perimetro del bacino in analisi, con le misure che avrebbe un cerchio di uguale superficie; quindi, più l'indice tende ad uno e maggiore sarà la compattezza dell'area studiata.\\
Al fine di applicare la relativa formula, è necessario conoscere i valori di perimetro del bacino (lunghezza dello spartiacque), e l'area della superficie dell'intera area di studio.\\
La formula dell'indicatore è:
\begin{equation}
    F = \frac{0.28 \cdot P}{\sqrt{A}} = \frac{0.28 \cdot 8.690}{2.01} = 1.21
    \label{gravelius}
\end{equation}

\subsection{Indice di compattezza $F_c$}
Proprio come per l'indice precedente, questa formula permette di comprendere la forma del bacino.\\
Al contrario del coefficiente precedente, che utilizza il valore del perimetro, questa formula necessita di conoscere la lunghezza del reticolo idrografico del bacino, oltre che alla superficie drenante interna allo spartiacque.\\
La formula dell'indicatore è:
\begin{equation}
    F_c = \frac{L}{\sqrt{A}} = \frac{5.290}{\sqrt{2.01}} = 1.42
    \label{compatezza}
\end{equation}

\subsection{Indice di Melton}
L'indice di Melton quantifica l'energia potenziale (gravitativa) del bacino.\\
La formula è applicabile conoscendo il dislivello di quota tra quella massima del bacino e quella dell'apice del cono di deiezione, oltre che l'area totale del bacino. 
\begin{equation}
    Me = \frac{\Delta H}{\sqrt{A}} = \frac{2084.6-1186}{\sqrt{2.01 \cdot 10^6}} = 0.634
    \label{melton}
\end{equation}
Questo fattore è importate per valutare i processi di trasporto solido che potrebbero avvenire; infatti, se questo valore è maggiore (o uguale) a 0.5, l'energia potenziale dei sedimenti potrebbe da dare luogo a fenomeni di trasporto, come per esempio le colate detritiche.

\subsection{Densità di drenaggio}
Questo indicatore, anche detto \textit{``drainage density"}, mette in relazione la lunghezza complessiva di tutti i rami del reticolo idrografico e l'area del bacino.\\
Tale valore è molto influenzato dalla quantità pluviometrica annua, dalle caratteristiche del suolo (litologia ed erodibilità) e dalla destinazione d'uso dell'area (copertura vegetale, impermeabilizzazione antropica,\dots).\\
La formula è:
\begin{equation}
    D_r = \frac{\Sigma L_i}{A} = \frac{5.290}{2.01} = 2.63 km^{-1}
    \label{drenaggio}
\end{equation}

\subsection{Indice di torrenzialità}
Questo indice rapporta il numero dei rami del reticolo con l'area totale del bacino idrografico.
La formula è:
\begin{equation}
I_t = \frac{N}{A} = \frac{16}{2.01} = 7.96 \hspace{3pt}\frac{n. segmenti}{km^2} 
\label{torrenzialità}
\end{equation}

\subsection{Quota media del bacino}
Questo valore quantifica l'altezza media del bacino rispetto alla sezione di chiusura, ponderando l'area.\\
Dal punto di vista geometrico, la quota media del bacino è il valore che, se introdotto nell'asse delle ordinate, crea un rettangolo di area uguale a quella sottesa alla curva ipsografica, aventi entrambi gli stessi valori in ascissa.\\
La formula è:
\begin{equation}
    H_m = h_m - h_0 = \frac{\Sigma (\bar{h}_i - h_0) \cdot A_i}{A}
    \label{eq:quota_media}
\end{equation}

I calcoli, svolti in un foglio di calcolo elettronico, sono qui sotto riportati.

\begin{table}[H]
    \begin{tabular}{cccccccc}
        \toprule
    Quote (m s.l.m.) & n° celle & $A_i$ (\unit{m^2}) & $A_i$ (\unit{km^2}) & $Hi_{min}$ & $Hi_{max}$ & $Hi_{media}$ & $A_i \cdot Hi_{media}$ \\
   \midrule
    1183-1200                 & 261      & 261     & 0.000261 & 1183   & 1200   & 1192     & 0.311         \\
    1200-1250                 & 4874     & 4874    & 0.004874 & 1200   & 1250   & 1225     & 5.971         \\
    1250-1300                 & 27605    & 27605   & 0.027605 & 1250   & 1300   & 1275     & 35.196        \\
    1300-1350                 & 57101    & 57101   & 0.057101 & 1300   & 1350   & 1325     & 75.659        \\
    1350-1400                 & 108630   & 108630  & 0.10863  & 1350   & 1400   & 1375     & 149.366       \\
    1400-1450                 & 138207   & 138207  & 0.138207 & 1400   & 1450   & 1425     & 196.944       \\
    1450-1500                 & 160541   & 160541  & 0.160541 & 1450   & 1500   & 1475     & 236.797       \\
    1500-1550                 & 164185   & 164185  & 0.164185 & 1500   & 1550   & 1525     & 250.381       \\
    1550-1600                 & 199048   & 199048  & 0.199048 & 1550   & 1600   & 1575     & 313.500       \\
    1600-1650                 & 178951   & 178951  & 0.178951 & 1600   & 1650   & 1625     & 290.794       \\
    1650-1700                 & 175437   & 175437  & 0.175437 & 1650   & 1700   & 1675     & 293.856       \\
    1700-1750                 & 158025   & 158025  & 0.158025 & 1700   & 1750   & 1725     & 272.592       \\
    1750-1800                 & 152704   & 152704  & 0.152704 & 1750   & 1800   & 1775     & 271.049       \\
    1800-1850                 & 150620   & 150620  & 0.15062  & 1800   & 1850   & 1825     & 274.881       \\
    1850-1900                 & 115773   & 115773  & 0.115773 & 1850   & 1900   & 1875     & 217.074       \\
    1900-1950                 & 114122   & 114122  & 0.114122 & 1900   & 1950   & 1925     & 219.684       \\
    1950-2000                 & 82232    & 82232   & 0.082232 & 1950   & 2000   & 1975     & 162.408       \\
    2000-2050                 & 18149    & 18149   & 0.018149 & 2000   & 2050   & 2025     & 36.752        \\
    2050-2085                 & 1317     & 1317    & 0.001317 & 2050   & 2085   & 2067     & 2.723        \\
    \bottomrule
\end{tabular}
    \end{table}
Dove:
\begin{itemize}
    \item Quote: fasce di quota, secondo le linee di livello ortometriche;
    \item n° celle: numero di celle del raster, contenute in ogni fascia di quota;
    \item $A_i$: area di territorio ricadente in ogni fascia di quota (sia in \unit{m^2} che in \unit{km^2});
    \item $Hi_{min}$ e $Hi_{max}$: quota minima e massima della fascia di quota;
    \item $Hi_{media}$: valore altimetrico medio per ogni fascia di quota;
    \item $A_i \cdot Hi_{media}$: altezza media di ogni fascia altimetrica, soppesata per la propria estensione.
\end{itemize}
Considerando che le celle hanno estensione 1 \unit{m^2}, l'area di ogni fascia di quota equivale (in \unit{m^2}) al proprio numero di quadrati della matrice.\\
Essendo che l'area totale del bacino è di 2.01 \unit{km^2}, la formula \eqref{eq:quota_media}, ricavata dal foglio di calcolo elettronico, restituisce il valore di 1646 \unit{m} sul livello del mare.

\subsection{Curva ipsometrica del bacino}
Al fine di valutare la condizione di dinamica temporale del bacino, risulta utile valutare le curve ipsometriche dello stesso.\\
Osservando l'andamento della curva che lega l'area del bacino e la quota ortometrica (siano queste dimensionali o adimensionali), è possibile capire se l'ara studiata è in una condizione giovanile (dove prevale l'erosione), maturità (equilibrio) o senilità.
\begin{figure}[H] \centering
          \includegraphics[scale=0.7]{immagini/curva_ipsografica_dimens.png}
        \caption{Curva ipsografica dimensionale del bacino. In ascissa è indicato il valore di area cumulata, ed in ordinata la relativa quota ortometrica.}
\label{curva_ipsometrica_dimensionale}
\end{figure}    
\begin{figure}[H] \centering
        \includegraphics[scale=0.8]{immagini/curva_ipsografica_adimens.png}
        \caption{Curva ipsografica adimensionale del bacino. Al contrario dell'altro grafico, i valori degli assi sono adimensionali, ovvero rapportati con i massimi della serie.}
\end{figure}
Mediante entrambi i grafici è possibile osservare come il bacino si trovi in uno stato di equilibrio.\\
Mediante la bisettrice, nel secondo grafico, si può apprezzare maggiormente come, verso monte del bacino, l'erosione sia stata maggiore rispetto che a valle.

\subsection{Pendenza media dei versanti del bacino $i_m$}
L'indicatore in questione evidenzia, in modo mediato, la pendenza dei versanti del bacino.\\
Per calcolare tale numero si utilizza il Metodo Harlvord-Horton, mediante la formula:
\begin{equation}
    i_m = \Delta h \cdot \frac{\Sigma l_i}{A}
\end{equation}
dove: 
\begin{itemize}
 \item $l_i$ è la lunghezza totale dell'isoipsa considerata e delimitata all'interno dell'area del bacino;
 \item $A$ è la superficie totale del bacino idrografico.
\end{itemize}

\subsection{Distribuzione della pendenza lungo il bacino}
Conoscendo la pendenza di ogni singola fascia di quota, è possibile raffigurare in un grafico la pendenza media di ogni sezione di bacino, avente quote altimetriche simili.
\begin{figure}[H] \centering
\includegraphics[scale=0.8]{immagini/istogramma_distribuzione_pendenze.png}
    \caption{Istogramma indicante le distribuzioni delle pendenze del bacino.}
\end{figure}

\subsection{Distribuzione delle aree del bacino}
Il seguente grafico indica come l'area del bacino idrografico venga suddivisa in base alla fascia altimetrica.
\begin{figure}[H] \centering
    \includegraphics[scale=0.8]{immagini/istogramma_distribuzione_aree.png}
    \caption{Istogramma indicante le distribuzioni delle aree del bacino.}
\end{figure}

\subsection{Profilo altimetrico del torrente}
Il profilo altimetrico del torrente (che nel nostro caso è il corso principale del bacino idrografico) indica l'andamento ortometrico, in funzione della propria lunghezza cumulata.
\begin{figure}[H] \centering
    \includegraphics[scale=0.8]{immagini/prof_coll_princ.png}
    \caption{Profilo longitudinale del collettore del bacino idrografico.}
\end{figure}
Il grafico è stato ricavato mediante l'utilizzo di un foglio elettronico di calcolo, utilizzando i dati provenienti da Qgis.\\
Inoltre, mediante tali dati è possibile ricavare la pendenza media del tratto principale del corso idraulico (28.94\%, oppure 16.14°) e la sua lunghezza totale (2401 \unit{m}).


\subsection{Caratteristiche del bacino idrografico}
Successivamente ad aver analizzato in modo completo il bacino idrografico, è possibile redigere una tabella riassuntiva.
\begin{table}[H] \centering
    \caption{\textcolor{red}{Caratteristiche del bacino idrografico  del torrente Sigismondi chiuso alla confluenza con il torrente Fersina.}}
    \label{tab:caratteristiche_bacino}
    \begin{tabular}{ cccc } 
    \toprule
    Superficie planimetrica & A &  $\left[\si{\kilo\square\meter}\right]$ & 2.01 \\ 
    Perimetro & P & [\unit{m}]        &      8690       \\ 
    Quota massima & $h_{max}$&  [m s.m.]       &    2084.6     \\
    Quota della sezione di chiusura & $h_0$ & [m s.m.] &    1186      \\ 
    Quota apice del conoide &$h_{ap}$& [m s.m.]& 1186\\ 
    Quota media& $H_m$ & [m s.m.]& 1646    \\ 
    Rilievo del bacino:& $h_{max} - h_o$ & \unit{m} &   901 \\ 
    Lunghezza del reticolo idrografico& $L_r$& \unit{m} & 5294,1 \\ 
    Lunghezza del collettore principale& $L_{coll}$& \unit{m} & 2401 \\ 
    Pendenza media del bacino& $i_m$ & $\left[\%\right]$ & 54.19    \\ 
    Pendenza media del bacino& $i_m$& $\left[ ^\circ \right]$ & 28.45   \\ 
    Coefficiente di forma di Gravelius& F& $\left[0.28 \cdot P/A^{0.5} \right]$ & 1.21\\  
    Indice di compattezza &$F_c$  & $ \left[L/A^{0.5}\right]$ & 1.42\\ 
    Numero di Melton& & $\left[-\right]$ & 0.634\\ 
    Densità di drenaggio &$D_r$& $\left[\si{\km^{-1}}\right]$& 2.63 \\ 
    $N_1$& & & 12 \\ 
    Indice di torrenzialità& & [\unit{segm/km^2}] & 7.96 \\  
    Ordine di bacino& $k$ & $\left[-\right]$ & 3 \\ 
    Rapporto di biforcazione medio& $R_b$ & $\left[-\right]$ &  3.5 \\  
    Pendenza media del collettore principale & & [\unit{m/m}] & 0.289\\  
    \bottomrule
\end{tabular}
\end{table}