\section{Elaborazione statistico-probabilistica delle pioggie intense della stazione di riferimento per il bacino in esame (analisi degli afflussi)}
In questo capitolo della relazione si condurrà un'analisi statistico-probabilistica degli eventi pluviometrici del bacino. In particolare, si utilizzerà il metodo dei momenti di Gumbel.\\
Successivamente, verrà effettuata la verifica dell'adattamento della serie di valori pluviometrici alla distribuzione (di Gumbel nel nostro caso) prescelta. Per fare ciò, si utilizzerà il test di Matalas.\\
Infine, si svolgeranno i calcoli per determinare la Linea Segnalatrice di Possibilità Pluviometrica (LSPP).

\subsection{Verifica preliminare mediante \textit{Plotting position}}
Prima di svolgere l'adattamento della serie pluviometrica misurata a quella degli estremi di Gumbel, occorre valutare preliminarmente se tali valori si adattano correttamente lungo la retta di distribuzione.\\
In pratica, viene verificata se l'altezza di precipitazione $h$ segue linearmente la variabile di appoggio $y$.\\
La procedura 

\subsection{Regolarizzazione secondo Gumbel (metodo dei momenti)}
Al fine di interpretare la serie storica degli eventi pluviometrici, occorre adattare i valori ad una certa distribuzione di probabilità.\\
I dati della serie campionaria devono: 
\begin{itemize}
    \item riguardare i massimi annuali di precipitazioni di un certo periodo (almeno 30 anni);
    \item essere \textit{casuali, omogenei, indipendenti e stazionari}.
\end{itemize}
