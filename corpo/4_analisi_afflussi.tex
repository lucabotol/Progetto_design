\section{Elaborazione statistico-probabilistica delle pioggie intense della stazione di riferimento per il bacino in esame (analisi degli afflussi)}
In questo capitolo della relazione si condurrà un'analisi statistico-probabilistica degli eventi pluviometrici del bacino. In particolare, si utilizzerà la distribuzione di probabilità dei valori estremi (legge di Gumbel).\\
Successivamente, verrà effettuata la verifica dell'adattamento della serie di valori pluviometrici alla distribuzione, facendo ricorso al test di Matalas.\\
Infine, si svolgeranno i calcoli per determinare la Linea Segnalatrice di Possibilità Pluviometrica (LSPP).\\
Essendo che si andrà a svolgere i calcoli considerando solamente i massimi annuali di precipitazione, occorre che tali valori debbano: 
\begin{itemize}
    \item essere relativi ad un certo periodo di tempo (almeno 30 anni);
    \item essere \textit{casuali, omogenei, indipendenti e stazionari}.
\end{itemize} 

\subsection{Verifica preliminare mediante \textit{Plotting position}}
Prima di svolgere l'adattamento della serie pluviometrica misurata a quella degli estremi di Gumbel, occorre valutare se tali valori si adattano correttamente lungo la retta di distribuzione teorica.\\
Ad ogni misura della serie, posta in ordine crescente, viene attribuito un valore a seconda della propria posizione nel campione, secondo il metodo di Weibull $(P_{em}= \frac{i}{N+1})$ o Hazen $\left(P_{em}=\frac{i-0.5}{N}\right)$. Entrambi questi parametri correlano l'evento di precipitazione con la probabilità di non superamento dell'evento (anche indicato come $P$).\\
Utilizzando il valore ricavato da uno di questi metodi, viene calcolata una variabile d'appoggio ($y$), secondo la formula $y = -ln(-ln(P))$, essendo che $P(x)= e ^{-e^ {-y}}$.\\
\noindent Successivamente, utilizzando i parametri $y$ (appena calcolato), $\alpha$ e $u$, quest'ultimi basati sul campione, si riesce a ricavare l'altezza di precipitazione dell'evento, regolarizzata secondo la distribuzione di Gumbel, secondo la formula inversa di $y = \alpha (h - u)$.
\begin{equation}
    \alpha = \left(\frac{1.283}{\sigma(h)}\right)
\end{equation}
\begin{equation}
    u = \left( h_m - \frac{0.5772}{\alpha} \right)
\end{equation}
\begin{equation}
    h = \frac{y}{\alpha}+u
    \label{h_tr}
\end{equation}
Al fine di verificare visivamente la bontà degli allineamenti delle distribuzioni della serie osservata e della serie attesa (teorica), occorre creare un grafico dove, in ascissa si pone il valore del parametro $y$ ed in ordinata si pongono i valori delle relative precipitazioni.\\
Maggiore è l'allineamento tra le due distribuzioni e migliore sarà la distribuzione della serie osservata.

\subsection{Calcolo delle altezze di pioggia per il $T_r$ di riferimento}
Successivamente ad aver verificato che l'allineamento tra le due distribuzioni sia apprezzabile, occorre svolgere il calcolo delle altezze di pioggia, dati dei tempi di ritorno.\\
Essendo che $y = -ln(-ln(P))$ e $P = 1-\frac{1}{T_r}$, allora 
\begin{equation}
y = -ln(-ln(1-\frac{1}{T_r}))
\end{equation}
Partendo da questo calcolo della variabile d'appoggio, è possibile ricavare l'altezza di precipitazione mediante la formula \ref{h_tr}.


