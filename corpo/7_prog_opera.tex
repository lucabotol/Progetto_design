\section{Progettazione di un'opera di consolidamento tipo}
Per iniziare il dimensionamento delle singole opere di consolidamento, occorre determinare il numero complessivo degli elementi da ereggere.\\
Opere troppo elevate generano un impatto significativo sull'ambiente, poiché tendono a deturpare il paesaggio e la risalita della fauna ittica. Opere di dimensioni troppo ridotte comportano un numero eccessivo di elementi, posti a distanza molto ravvicinata.\\
L'esperienza del professore ci consiglia di imporre un numero di briglie pari a 9, con altezza di 2.63 m.\\
I calcoli verranno svolti basandosi sul diagramma di deflussi per precipitazioni con intensità a blocchi alternati (mediante metodo cinematico), comprendenti anche il trasporto solido, per tempi di ritorno pari a 200 anni. Tale valore di portata totale è di 9.25 $m^3/s$.\\
I parametri caratteristici del tratto di torrente da sottoporre a sistemazione sono presenti nel tabella riportata precedentemente \ref{inf_morf_coll_princ}.\\
Per quanto riguarda i parametri relativi al numero di briglie, si fa riferimento al grafico precedente \ref{opzioni_numero_altezza_briglie}.
\subsection{Gaveta a profilo trapezoidale}
\subsubsection{Dimensionamento iniziale}
Si impone come larghezza inferiore della gaveta la metà della larghezza del torrente, ovvero 4 metri.\\
Successivamente ad essersi calcolati la larghezza inferiore, occorre calcolare la profondità del flusso d'acqua passante per la sezione della gaveta. Come valore di prima approssimazione si applica la formula:
\begin{equation}
    h = 0.7 \cdot q ^{2/3}
\end{equation}