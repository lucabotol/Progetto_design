\section{Progettazione di un'opera di consolidamento tipo}
Per iniziare il dimensionamento delle singole opere di consolidamento, occorre determinare il numero complessivo degli elementi da ereggere.\\
Opere troppo elevate generano un impatto significativo sull'ambiente, poiché tendono a deturpare il paesaggio e la risalita della fauna ittica. Opere di dimensioni troppo ridotte comportano un numero eccessivo di elementi, posti a distanza molto ravvicinata.\\
L'esperienza del professore ci consiglia di imporre un numero di briglie pari a 9, con altezza di 2.63 m.\\
I calcoli verranno svolti basandosi sul diagramma di deflussi per precipitazioni con intensità a blocchi alternati (mediante metodo cinematico), comprendenti anche il trasporto solido, per tempi di ritorno pari a 200 anni. Tale valore di portata totale è di 9.25 $m^3/s$.\\
I parametri caratteristici del tratto di torrente da sottoporre a sistemazione sono presenti nel tabella riportata precedentemente \ref{inf_morf_coll_princ}.\\
Per quanto riguarda i parametri relativi al numero di briglie, si fa riferimento al grafico precedente \ref{opzioni_numero_altezza_briglie}.
\subsection{Gaveta a profilo trapezoidale}
\subsubsection{Dimensionamento iniziale}
Si impone come larghezza inferiore della gaveta la metà della larghezza del torrente, ovvero 4 metri.\\
Successivamente ad essersi calcolati la larghezza inferiore, occorre calcolare la profondità del flusso d'acqua passante per la sezione della gaveta. Come valore di prima approssimazione si applica la formula:
\begin{equation}
    h = 0.7 \cdot q ^{2/3}
\end{equation}
Nel nostro caso, la formula diventa:
\begin{equation}
    h = 0.7 \cdot \left( \frac{9.25}{4} \right)^{2/3}
\end{equation}
Che restituisce il risultato di 1.22 m.\\
Successivamente ad aver stimato la profondità di massima, occorre ricavare la profondità corretta, utilizzando in modo iterativo la formula di Belangér (ricavandosi la medesima portata di progetto). Tale formula è la seguente:
\begin{equation}
Q = 1.705 \cdot h^{3/2} \left[L + \frac{2}{5}\cdot h \cdot \left(\frac{1}{\tan{\alpha}} + \frac{1}{\tan{\beta}}\right) \right]
\end{equation}
Si è scelto di imporre i valori $\alpha$ e $\beta$ (ovvero la pendenza delle ali laterali), pari a 60$^\circ$.\\
Dopo diversi tentativi, l'altezza che genera una portata totale di 9.25 $m^3/s$, in funzione della geometria della gaveta, è di 1.13 m.\\
Si è scelto di imporre un franco idraulico di 13 cm, in modo da portare l'altezza della sezione della gaveta a 1.30 m.\\
Avendo calcolato l'altezza della sezione trapezoidale e conoscendo la pendenza delle ali, è possibile calcolare la larghezza superiore della gaveta:
\begin{equation}
    L_{gav.sup} = L_{gav.inf} + 2 \cdot \frac{h}{\tan \alpha} 
\end{equation}
I calcoli reali portano ad una larghezza superiore della briglia di 5.50 m, essendo che la larghezza di una singola ala è di 1.25 m.\\
La larghezza media della sezione trapezoidale della briglia avviene calcolando la media tra la larghezza inferiore e superiore, che in questo caso è 4.75 m.\\
Lo spessore del coronamento della briglia (s) viene stimato utilizzando tre metodi diversi:
\begin{itemize}
    \item metodo di Zoli: $0.7 + 0.1 \cdot Z$ $\rightarrow$ 0.96 m;
    \item verifica allo scorrimento: $0.7 \cdot h$ $\rightarrow$ 0.91 m;
    \item metodo di Romiti: $0.8 + 2 \cdot D_{84}$ $\rightarrow$ 1.23 m.
\end{itemize}
Essendo che i tre risultati non si discostano molto tra di loro, viene scelto il valore più cautelativo, imponendo la larghezza del coronamento pari a 1.2 m.\\
Il dimensionamento della base del corpo briglia può avvenire applicando due formule empiriche:
\begin{itemize}
    \item metodo di Zoli: implica l'utilizzo del specifico grafico a doppia entrata $\rightarrow$ 1.80 m;
    \item metodo di Romiti: $z \sqrt{\frac{z+3 \cdot h-s^2/z}{z+h+4.55}}$ $\rightarrow$ 2.13 m.
\end{itemize}
Anche per questo caso, essendo che i valori ricavati non si discostano molto tra di loro, si scegliere di porsi nelle condizioni maggiormente cautelative, andando ad imporre una larghezza di base del corpo briglia di 2.10 m.\\
La profondità della fondamenta $(z_f)$ della briglia viene calcolata in funzione della pool erosiva che si genera a valle dell'opera (l'argomento verrà ripreso nel capitolo successivo inerente alla controbriglia), o in funzione della sola portata totale passante per la gaveta. Tale profondità può essere calcolata mediante due formule:
\begin{itemize}
    \item $z_f > 0.6 \cdot t$ $\rightarrow$ 0.63 m;
    \item $z_f > 0.15 \cdot (z+h)$ $\rightarrow$ 0.56 m
\end{itemize}
Viene scelto il valore maggiore tra i due, arrotondando a 0.65 m, per motivi pratici e di sicurezza.\\
Infine, l'ultima geometria da calcolare per quanto riguarda la briglia è inerente alla lunghezza degli sporti, sia di monte che di valle. Tale valore dev'essere minore (o uguale) a $0.7 \cdot z_f$. In questo caso, la lunghezza massima dello sporto della fondamenta è 0.45 m, ed in questo caso si impone lo stesso valore per la progettazione.\\
E' possibile che lo sporto di valle e di monte non abbiano la stessa lunghezza. 

\subsubsection{Verifiche tradizionali della briglia}
Prima di iniziare le verifiche dimensionali tradizionali della briglia, risulta utile riportare le misure calcolate durante il dimensionamento di massima ed i coefficienti necessari da attribuire alle forze e momenti.
\begin{table}[H] \centering
    \caption{\textcolor{red}{Valori dimensionali e coefficienti fisici necessari per effettuare la procedura di verifica della geometria della briglia.}}
    \begin{tabular}{cc}
    \toprule
    DATI                          &       \\
    \midrule
    peso sp. acqua ($N/m^3$)         & 10000 \\
    peso sp. mat. ($N/m^3$)          & 24000 \\
    altezza gaveta: h (m)         & 1.130 \\
    altezza corpo: z(m)           & 2.63  \\
    coronamento: s (m)            & 1.20  \\
    base corpo: b (m)              & 2.10  \\
    base fondazione: Bf (m)       & 2.40  \\
    altezza fondazione: zf (m)    & 0.65  \\
    sporto monte:sm (m)            & 0.30  \\
    sporto valle: sv (m)           & 0.00  \\
    coeff. attrito: fmur           & 0.7   \\
    coeff. attrito terreno: fterr & 0.45  \\
    coeff. riduz.sottospinta.: m  & 0.20  \\
    \bottomrule
    \end{tabular}
\end{table}