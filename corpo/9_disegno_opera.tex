\section{Disegno in scala dell'opera progettata}
I disegni tecnici, allegati alla relazione, rappresentano sia le singole opere (briglia e controbriglia) e sia la sistemazione complessiva del tratto di torrente.\\
Al fine di rendere l'opera meno impattante dal punto di vista paesaggistico, e maggiormente realistica, è stato ipotizzato che la superficie dell'opera abbia un rivestimento esterno in pietra a faccia vista.\\
Essendo che in fase di dimensionamento le due ali laterali sono state ipotizzate orizzontali, di conseguenza il disegno tecnico deve rappresentare allo stesso modo le sezioni dell'opera.\\
Generalmente però, le ali della briglia hanno una pendenza all'incirca del 10\% rispetto all'orizzontale, in modo da permettere il passaggio di portate con tempi di ritorno superiori a quello di progetto. In tale modo, si scongiura anche il rischio di eventuali aggiramenti del torrente rispetto alla briglia di consolidamento.