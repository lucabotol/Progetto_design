\section{Risultati}
In questo capitolo verranno esposti i risultati dei calcoli, relativi all'altezza di precipitazione per un assegnato tempo di ritorno, per la portata di picco e per il dimensionamento del canale.
\subsection{Altezza di precipitazione}
\begin{table}[H] \centering
\begin{tabular}{cc}
                      & \textbf{h (mm)}       \\
\toprule
\textbf{durata (ore)} & \textbf{Tr = 50 anni} \\
\midrule
1                     & 48.5                  \\
3                     & 57.2                  \\
6                     & 63.8                  \\
12                    & 90.9                  \\
24                    & 137.7              \\
\bottomrule
\end{tabular}
\end{table}

\subsection{Portata di picco}
Per completezza si ripete questo valore, anche se è già stato anticipato al paragrafo \ref{portatadipicco}.
La portata di picco del bacino è 26.82 $\frac{m^3}{s}$.

\subsection{Profondità di moto uniforme}
I risultati parziali, e quello finale, sono riportati nella seguente tabella:
\begin{table}[H] \centering
\begin{tabular}{cccc}
\toprule
$y_0$ & Rh   & y    & $\Delta H$ (\%) \\
\midrule
2.83 & 0.98 & 5.73 & 102                          \\
5.73 & 1.19 & 5.04 & 13.7                         \\
5.04 & 1.16 & 5.13 & 1.78                          \\
\bottomrule
\end{tabular}
\end{table}
Quindi, la profondità finale del canale, per cui avviene la condizione di moto critico, è 5.13 m.
\subsection{Energia specifica e larghezza limite}
Come esposto precedentemente, l'energia specifica del canale è 5.285 m, mentre la larghezza limite è di 1.702 m.
%%%%%%%%%%%%%%%%%%%%%%%%%%%%%%%%%%%%%%%%%%%%%%%%%%%%%%%%%%%%%%%%%%%%%%%%%%%
